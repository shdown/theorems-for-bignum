\documentclass[a4paper,12pt]{article}
\usepackage{standalone}

\usepackage[utf8]{inputenc}
\usepackage[english]{babel}
\usepackage{amssymb}
\usepackage{amsmath}
\usepackage{amsthm}
\usepackage{mathrsfs}

\makeatletter

\newtheorem{theorem}{Theorem}
\newtheorem{lemma}{Lemma}

\newcommand{\floor}[1] {\lfloor #1 \rfloor}

\title{Theorems for long division and root extraction}
\author{Viktor Krapivensky}
\date{%
    \today
}

\begin{document}

\maketitle

\section{Notations}

$\mathbb{N} = \{0, 1, 2, \ldots\}.$

\section{Three-by-two theorem}

This theorem is extremely useful for basecase division.

\begin{theorem}[three-by-two]
    Fix $B \in \mathbb{N},$ $B > 1,$ the base of our positional number system.
    Fix also numbers $x \in \mathbb{R},$ $y \in \mathbb{R}$ such that:
    \begin{itemize}
      \item $0 \le x < B^3;$
      \item $B \le y < B^2;$
      \item $x/y < B.$
    \end{itemize}
    Define:
    \begin{itemize}
      \item $q = \floor{\frac{x}{y}},$ the true quotient;
      \item $q_e = \floor{\frac{\floor x}{\floor y}},$ our estimate of the quotient.
    \end{itemize}
    Then either $(q = q_e)$ or $(q = q_e - 1).$
\end{theorem}
\begin{proof}
    \begin{lemma}
        $q \le q_e.$
    \end{lemma}
    \begin{proof}
        Define $\delta = x - \floor{x};$
        note that $0 \le \delta < 1.$

        We have $\frac{x}{y} \le \frac{x}{\floor{y}} = \frac{\floor{x} + \delta}{\floor{y}}.$
        Then $q = \floor{\frac{x}{y}} \le \floor{\frac{\floor{x} + \delta}{\floor{y}}}.$

        We now want to prove $\floor{\frac{\floor{x} + \delta}{\floor{y}}} = \floor{\frac{\floor x}{\floor y}}.$
        Since $\delta < 1,$ for any integers $M, N, K,$ the following holds: $(M < KN) \implies ((M + \delta) < KN).$

        Substitute $M = \floor{x},$ $N = \floor{y},$ $K = \floor{\frac{M}{N}} + 1.$
    \end{proof}

    \begin{lemma}
    \label{lemma:floorx_bound}
        $\floor{x} < B (\floor{y} + 1).$
    \end{lemma}
    \begin{proof}
        $\floor{x} \le x < By < B(\floor{y} + 1).$
    \end{proof}

    Define now the following values:
    \begin{itemize}
        \item $u = \floor{x};$
        \item $v = \floor{y};$
        \item $q_{\max} = \frac{u+1}{v};$
        \item $q_{\min} = \frac{u}{v+1}.$
    \end{itemize}

    \begin{lemma}
    \label{lemma:bounds}
        The following bounds hold:
        \begin{enumerate}
            \item $q_{\max} - \frac{u}{v} \le \frac{1}{B};$
            \item $\frac{u}{v} - q_{\min} < 1.$
        \end{enumerate}
    \end{lemma}
    \begin{proof}
        \begin{enumerate}
            \item $\frac{u+1}{v} - \frac{u}{v} = \frac{1}{v} \le \frac{1}{B};$

            \item $\frac{u}{v} - \frac{u}{v+1} = \frac{u}{v(v+1)}.$
            By lemma~\ref{lemma:floorx_bound}, $u < B(v+1),$ so
            $$\tfrac{u}{v} - q_{\min} < \tfrac{B(v+1)}{v(v+1)} = B/v \le 1.$$
        \end{enumerate}
    \end{proof}

    \begin{lemma}
        $q \ge q_e - 1.$
    \end{lemma}
    \begin{proof}
        We have:
        \begin{itemize}
            \item $q_{\min} < \frac{x}{y} < q_{\max};$
            \item $q_{\min} < \frac{u}{v} < q_{\max}.$
        \end{itemize}
        Taking floor of both sides of these inequalities, we get:
        \begin{itemize}
            \item $\floor{q_{\min}} \le q \le \floor{q_{\max}};$
            \item $\floor{q_{\min}} \le q_e \le \floor{q_{\max}}.$
        \end{itemize}
        By lemma~\ref{lemma:bounds}, $q_{\max} - q_{\min} < 1 + \frac{1}{B} < 2.$ This means that
        $\floor{q_{\max}} - \floor{q_{\min}}$ is either 0, 1, or 2.
        We are only interested in the case of it being $2,$ as in other cases $(q \ge q_e-1)$ holds automatically.

        Suppose $\floor{q_{\max}} - \floor{q_{\min}} = 2$ and $q_e - q = 2.$ Then,
        \begin{itemize}
            \item $\floor{q_{\max}} = q_e = \floor{\frac{u}{v}},$ which implies $\frac{u}{v} \ge q_e = q + 2;$
            \item $\floor{q_{\min}} = q,$ which implies $q_{\min} < q + 1.$
        \end{itemize}
        Together, these statements imply $\frac{u}{v} - q_{\min} > 1,$ contradicting lemma~\ref{lemma:bounds}.
    \end{proof}
\end{proof}

\section{Approximation of inverse theorem}

This theorem is useful for calculating the inverse of a number with Netwon's method;
namely, it tells us how to find the initial approximation of the inverse.

\begin{theorem}[approximation of inverse]
    Fix $B \in \mathbb{N},$ $B > 1,$ the base of our positional number system.
    Fix then $n \in \mathbb{N},$ $n > 0,$ the number of words in our initial approximation.
    Fix a number $x \in \mathbb{R}$ such that $B^n \le x < B^{n+1}.$
    Define:
    \begin{itemize}
      \item $r = \frac{B^{2n}}{x},$ the true inverse (scaled up by $2n$ places);
      \item $r_e = \floor{\frac{B^{2n}}{\floor{x} + 1}},$ our estimate of the scaled-up inverse.
    \end{itemize}
    Then:
    \begin{itemize}
        \item $r-2 < r_e < r;$
        \item $B^{n-1} \le r_e < B^n.$
    \end{itemize}
\end{theorem}
\begin{proof}
    $r_e < r$ is trivial: we increased the denominator ($\floor{x}+1 > x$) and then took floor of the fraction.

    Define now the following values:
    \begin{itemize}
        \item $u = \floor{x};$
        \item $r' = \frac{B^{2n}}{u + 1}.$
    \end{itemize}
    Note that $r_e = \floor{r'}.$
    Then $r - r' = \frac{B^{2n}}{u(u+1)} \le \frac{B^{2n}}{B^{2n} + B^n} < 1.$
    Now we have $$r - r_e = (r - r') + (r' - \floor{r'}) < 1 + 1 = 2.$$

    We can prove $B^{n-1} \le r_e < B^n$ by substituting the maximum and minimum possible
    values of $\floor{x}+1$ into $r_e = \floor{\frac{B^{2n}}{\floor{x}+1}}.$
    The maximum possible value, $B^{n+1},$ gives us $r_e \ge B^{n-1};$
    and the minumum possible value, $B^{n}+1,$ gives us $r_e \le B^{n} - 1.$
\end{proof}

\section{Root extraction}

We are given $d \in \mathbb{N}$ and root order $n \in \mathbb{N}, n \ge 2.$
We need to calculate $\floor{\sqrt[n]{d}}.$

Define the ``true'' root $\xi = \sqrt[n]{d}.$
Using unmodified Newton's method,
we are going to iterate $y \mapsto \Phi(y),$ where
$$\Phi(y) = \tfrac{1}{n} \big( \tfrac{d}{y^{n-1}} + (n-1) \cdot y \big).$$

If $y = \xi \cdot \delta,$ then $\Phi(y) = \xi \cdot \varphi(\delta),$
where $$\varphi(x) = \frac{1 + (n-1)x^n}{n x^{n-1}}.$$

\begin{theorem}[icky]
    $1 < \varphi(x) < x$ for $x > 1.$
\end{theorem}
\begin{proof}

We have
$$\varphi(x) < x \Leftrightarrow 1 + (n-1)x^n < n x^n \Leftrightarrow 1 - x^n < 0.$$
Now we will prove $\varphi(x) > 1.$

$\frac{1 + (n-1)x^n}{n x^{n-1}} > 1 \Leftrightarrow (1 + \varepsilon)^{n-1}(\varepsilon(n-1) - 1) > -1,$
where $\varepsilon = x - 1 > 0.$

Substituting $\lambda = \varepsilon(n-1) > 0$ and $m = n-1,$ we get
$$(1 + \tfrac{\lambda}{m})^m (\lambda - 1) > -1.$$

The sequence $E_m = (1 + \tfrac{\lambda}{m})^m$ increases monotonically for $\lambda > 0,$
and $\lim\limits_{m \to \infty} E_m = e^{\lambda}.$
This means $0 < (1 + \tfrac{\lambda}{m})^m < e;$ we are now going to prove
$$e^{\lambda} (\lambda - 1) > -1$$ for $\lambda > 0.$

The derivative $\frac{\mathrm{d}}{\mathrm{d} \lambda} e^{\lambda}(\lambda - 1) = e^{\lambda} \cdot \lambda$
is positive for $\lambda > 0;$ and $e^\lambda (\lambda - 1) = -1$ for $\lambda = 0.$

\end{proof}

\begin{theorem}[root extraction]
    Consider now the following process:
    we start with an arbitrary integer $y_0 \ge \xi,$ and then,
    while $y_i > \xi,$ put $y_{i+1} = \floor{\Phi(y_i)}.$

    This process will terminate at some finite step $k \ge 0$ with $y_k = \floor{\xi}.$
\end{theorem}
\begin{proof}

Note that $\Phi(y) = \xi \varphi(y / \xi).$

\begin{lemma}
    $\floor{\Phi(y_i)} < y_i$ for any integer $y_i > \xi.$
\end{lemma}
\begin{proof}
    $\floor{\Phi(y_i)} \le \Phi(y_i) < y_i.$
\end{proof}

\begin{lemma}
    If, for some integer $y_i,$ we have $y_i > \xi$ and $y_{i+1} = \floor{\Phi(y_i)} \le \xi,$ then
    $y_{i+1} = \floor{\xi}.$
\end{lemma}
\begin{proof}
    We have $y_{i+1} = \floor{\Phi(y_i)} \le \xi < \Phi(y_i).$
\end{proof}

\end{proof}

Note that $(y > \xi) \Leftrightarrow (y^n > d),$
and $$\floor{\Phi(y)} = \floor{ ( \floor{ d / {y}^{n-1} } + (n-1) \cdot y ) / n }.$$

\end{document}
